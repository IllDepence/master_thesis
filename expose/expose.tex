% THIS IS SIGPROC-SP.TEX - VERSION 3.1
% WORKS WITH V3.2SP OF ACM_PROC_ARTICLE-SP.CLS
% APRIL 2009
%
% It is an example file showing how to use the 'acm_proc_article-sp.cls' V3.2SP
% LaTeX2e document class file for Conference Proceedings submissions.
% ----------------------------------------------------------------------------------------------------------------
% This .tex file (and associated .cls V3.2SP) *DOES NOT* produce:
%       1) The Permission Statement
%       2) The Conference (location) Info information
%       3) The Copyright Line with ACM data
%       4) Page numbering
% ---------------------------------------------------------------------------------------------------------------
% It is an example which *does* use the .bib file (from which the .bbl file
% is produced).
% REMEMBER HOWEVER: After having produced the .bbl file,
% and prior to final submission,
% you need to 'insert'  your .bbl file into your source .tex file so as to provide
% ONE 'self-contained' source file.
%
% Questions regarding SIGS should be sent to
% Adrienne Griscti ---> griscti@acm.org
%
% Questions/suggestions regarding the guidelines, .tex and .cls files, etc. to
% Gerald Murray ---> murray@hq.acm.org
%
% For tracking purposes - this is V3.1SP - APRIL 2009

\documentclass{proseminar}
\usepackage[hidelinks]{hyperref}
\usepackage[utf8]{inputenc}
\usepackage[T1]{fontenc}
\usepackage{graphicx}

\begin{document}

\conferenceinfo{Albert-Ludwigs Universit\"at Freiburg\\Technische Fakult\"at, Institut f\"ur Informatik\\Lehrstuhl f\"ur Datenbanken \& Informationssysteme}{}

\title{Expos\'e}
\subtitle{outlining a Master Thesis on:\\Semantic approaches to scientific citation recommendation (tentative title)}

\numberofauthors{1}
\author{Tarek Saier\\Reviewer: Prof. Dr. Georg Lausen\\Advisor: Dr.-Ing. Michael Färber}

\maketitle

\section{Introduction}
This expos\'e will outline a prospective Master Thesis in the area of scientific citation recommendation and argue for its value. The approach will encompass the creation of a dataset and development of supervised learning methods with a focus on semantic analysis of citation contexts. Evaluation of the resulting implementation will follow the most prevalent methods in the field.

The remainder of this document is structured as follows. Section 2 will provide some theoretical background on relevant areas and give a quick overview of related work. A detailed description of the planned methodology and approach will be given in section 3; followed in section 4 by an outline of the planned evaluation. Section 5 and 6 conclude the expos\'e by listing the foreseen contributions of the Thesis and a proposed schedule.

\section{Background}
The goal of citation recommendation is to provide adequate citations to a given input text. This can involve evaluating whether or not a given input text includes parts that are suitable to add citations to in the first place. Citations recommended can be given as just one citation for a given section of or position in an input text, or a ranked list of multiple possible citations. Another distinction can be made concerning the granularity of text that a citation is recommended for. This can range from a complete document to a specific point within a string of text. There are also approaches where so called citation markers---annotations in the text that show where a citation is to be placed---are given for the input text. In such a case the evaluation whether or not a citation should be recommended as well as the decision where exactly to put a citation are not necessary. In an ideal case, citation recommendation can even involve evaluating cited documents in terms of their quality.

Given there are a lot of dimensions along which approaches can differ, the following sub section will first explain relevant terminology and then give an overview of these distinguishing dimensions. This will enable a more easily understandable overview of related work. (add citation)

\subsection{Terminology}
foo

\subsection{Dimensions}
bar

\subsection{Related work}
- survey stuff (take more narrow look)\\
- wiki page news article papers (for entitiy stuff)

\section{Methodology and approach}
foo bar\\
MAG\cite{Paszcza2016}\cite{Herrmannova2016}\cite{Hug2017}\cite{Sinha2015}\\
entity\cite{Mishra2016}\\
claim\cite{Levy2014}\\
argument\cite{Goudas2014}

data sets that were considered and why (benefits, drawbacks, ... (cite accodingly))

MAG start and arXiv start scenario (see wiki)

details of arXiv processing, challenges, etc. (MAG for evaluation where citation marker position is not relevant)

\section{Evaluation}
foo bar

\section{Contributions}
- apparently semantic stuff not very explored (cite survey if possible, look at tables)  
- creation of another nice (exact citation markers, large citation context, etc.) dataset like gold standard paper\cite{Faerber2018}  
- a nice dataset like gold standard paper\cite{Faerber2018} but not restricted to CS domain

\section{Schedule}
\begin{table*}
\centering
% \caption{Schedule}
% \hphantom{ }
\begin{tabular}{|p{2.5cm}|p{6cm}|p{4cm}|} \hline
Time frame&Task&Results\\ \hline
Oct 1 -- Oct 21 & Develop mechanism to generate dataset with citation markers from arXiv source dump & Dataset boilerplate (i.e. with citation markers but no semantic annotation)\\ \hline
Oct 22 -- Oct 28 & Write expos\'e  & Thesis approval\\ \hline
Oct 29 -- Nov 04 & Addd entity annotations to dataset & Dataset usable for supervised learning\\ \hline
Nov 05 -- Nov 18 & Develop entity based recommendation approach & -\\ \hline
Nov 19 -- Nov 25 & Addd claim annotations to dataset & -\\ \hline
Nov 26 -- Dec 09 & Develop claim based recommendation approach & -\\ \hline
Dec 10 -- Dec 22 & Coordination with simultaneous tangential theses and integration into CiteRec system & -\\ \hline
Dec 23 -- Jan 06 & break/buffer & -\\ \hline
Jan 07 -- Jan 13 & Addd argument annotations to dataset & -\\ \hline
Jan 14 -- Jan 27 & Develop argument based recommendation approach and start offline evaluation & -\\ \hline
Jan 28 -- Feb 10 & Offline evaluation & -\\ \hline
Feb 11 -- Feb 24 & Online evaluation & -\\ \hline
Feb 25 -- Mar 17 & Thesis writing & -\\ \hline
Mar 18 -- Mar 31 & buffer/paper writing & -\\ \hline\end{tabular}
\end{table*}

% \section{}
% \subsection{}
% \subsubsection{}
% \footnote{}
% \begin{math}\lim_{n\rightarrow \infty}x=0\end{math}
% \begin{equation}\lim_{n\rightarrow \infty}x=0\end{equation}
% \begin{displaymath}\sum_{i=0}^{\infty} x + 1\end{displaymath}
% \begin{table}
% \centering
% \caption{Frequency of Special Characters}
% \begin{tabular}{|c|c|l|} \hline
% Non-English or Math&Frequency&Comments\\ \hline
% \O & 1 in 1,000& For Swedish names\\ \hline
% $\pi$ & 1 in 5& Common in math\\ \hline
% \$ & 4 in 5 & Used in business\\ \hline
% $\Psi^2_1$ & 1 in 40,000& Unexplained usage\\
% \hline\end{tabular}
% \end{table}
%
% \begin{table*}
% \centering
% \caption{Some Typical Commands}
% \begin{tabular}{|c|c|l|} \hline
% Command&A Number&Comments\\ \hline
% \texttt{{\char'134}alignauthor} & 100& Author alignment\\ \hline
% \texttt{{\char'134}numberofauthors}& 200& Author enumeration\\ \hline
% \texttt{{\char'134}table}& 300 & For tables\\ \hline
% \texttt{{\char'134}table*}& 400& For wider tables\\ \hline\end{tabular}
% \end{table*}
%
% \begin{figure}
% \centering
% \epsfig{file=fly.eps}
% \caption{A sample black and white graphic (.eps format).}
% \end{figure}
%
% \newtheorem{theorem}{Theorem}
% \begin{theorem}
% Let $f$ be continuous on $[a,b]$.  If $G$ is
% an antiderivative for $f$ on $[a,b]$, then
% \begin{displaymath}\int^b_af(t)dt = G(b) - G(a).\end{displaymath}
% \end{theorem}
%
% \begin{figure*}
% \centering
% \epsfig{file=flies.eps}
% \caption{A sample black and white graphic (.eps format)
% that needs to span two columns of text.}
% \end{figure*}

\bibliographystyle{abbrv}
\bibliography{../paper}  % bibliography.bib is the name of the Bibliography in this case

% remember to run:
% latex bibtex latex latex
% to resolve all references
%
% ACM needs 'a single self-contained file'!

\balancecolumns
\end{document}
