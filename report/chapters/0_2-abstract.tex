\chapter*{Abstract}
New research is being published at a rate, at which it is %
%practically
infeasible for many scholars, to read and assess everything possibly %
%/that could possibly be
relevant to their work. %
In pursuit of a remedy, efforts towards automated processing of publications, like semantic modelling of papers to facilitate their digital handling, and the development of information filtering systems, are an active area of research. %
% maybe add a sentence here, leading from research in this
% area *in general* to citation recommendation
In this thesis, we investigate the semantic modelling of citation contexts for the purpose of citation recommendation. For this, we generate a large data set with accurate citation information from \LaTeX{} sources of publications on arXiv.org. Using this data set, we develop semantic recommendation models based on entities and claim structures. To assess the effectiveness and conceptual soundness of our models, we perform a large offline evaluation on our own as well as several established data sets and furthermore conduct a user study. Our findings show, that the models can outperform a non-semantic baseline model and do, indeed, capture the kind of information they're conceptualized for.
%In evaluating our models on several data sets and through a user study, we show that they can outperform a non-semantic baseline and are especially well suited for the recommendation of the types of citations they're conceptualized for.

%Citation recommendation systems, a possible remedy, 
%Researchers spent a considerable amount of time identifying publications that are worthwhile reading and appropriate to reference. The development of systems to aid in these tasks is an active area of research.
%Citations are the means by which scholars relate their research to existing work, 

\chapter*{Zusammenfassung}
\begin{otherlanguage}{ngerman}
Angesichts der Menge und Frequenz neuer wissenschaftlicher Publikationen, ist es für viele Forschende nicht praktikabel, jede Veröffentlichung zu lesen, die potentiell relevant für die eigene Arbeit ist. Dieser Form der Informationsüberflutung entgegengesetzt stehen Bestrebungen zur automatisierten Verarbeitung akademischer Arbeiten, wie die semantische Modellierung wissenschaftlicher Publikationen zur Erleichterung deren maschineller Handhabung, sowie die Entwicklung akademischer Empfehlungsdienste. In dieser Thesis untersuchen wir die semantische Modellierung von Zitierkontexten zum Zwecke der Zitierempfehlung. Hierfür generieren wir einen großen Datensatz mit akkurater bibliographischer Vernetzung aus \LaTeX{}-Quelldaten von Publikationen auf arXiv.org. Mithilfe dieses Datensatzes entwickeln wir semantische Modelle von Zitierkontexten, basierend auf Entitäten und Behauptungsstrukturen. Zur Prüfung unserer Modelle auf Eignung für die Zitierempfehlung und konzeptionelle Richtigkeit, führen wir eine umfassende Offline-Evaluation anhand unseres eigenen und anderer Datensätze, sowie eine Nutzerstudie durch. Unsere Ergebnisse zeigen, dass die Modelle bessere Ergebnisse als eine nicht-semantische Referenzmethode erzielen können, und in der Tat die Informationen erfassen, für die sie konzipiert wurden.
\end{otherlanguage}

% Die Publikation neuer wissenschaftlicher Arbeiten findet in einer Menge und Frequenz statt, die es vielen Forschern impraktikabel gestaltet, jede für die eigene Arbeit potentiell relevante Veröffentlichung zu lesen und zu beurteilen. 
% Jede, für die eigene Forschung potentiell relevante Veröffentlichung, zu lesen und zu beurteilen, ist für viele Forscher angesichts der Menge und Frequenz neuer wissenschaftlicher Publikationen nicht praktikabel.
% Die Menge des Zuwachses wissenschaftlicher Publikationen 
% Die Menge neuer wissenschaftlicher Publikationen 
% Wissenschaftliche Publikationen werden 
% Die Menge, in der wissenschaftliche Publik
% Die Zahl neuer wissenschaftlicher Publikationen, die innerhalb eines gegebene Zeitraumes veröffentlicht werden, 
% Forschungsberichte
% Der Zuwachs an neuen wissenschafftlichen Publikationen ist auf einem Level
% Die Publikation neuer wissenschaftlicher Arbeiten (/Werke/Papiere) geschieht in einer Menge und Frequenz, 
