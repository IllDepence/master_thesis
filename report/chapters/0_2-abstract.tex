\chapter*{Abstract}
New research is being published at a rate, at which it is %
%practically
infeasible for many scholars, to read and assess everything possibly %
%/that could possibly be
relevant to their work. %
In pursuit of a remedy, efforts towards automated processing of publications, like semantic modelling of papers to facilitate their digital handling, and the development of information filtering systems, are an active area of research. %
% maybe add a sentence here, leading from research in this
% area *in general* to citation recommendation
In this thesis, we investigate the semantic modelling of citation contexts for the purpose of citation recommendation. For this, we generate a large data set with accurate citation information from publications' \LaTeX{} sources on arXiv.org. Using this data set, we develop recommendation models based on entities and claim structures. To assess the effectiveness and conceptual soundness of our models, we perform a large offline evaluation on our own as well as several established data sets and furthermore conduct a user study. Our findings show that the models can outperform a non-semantic baseline model and do, indeed, capture the kind of information they're conceptualized for.
%In evaluating our models on several data sets and through a user study, we show that they can outperform a non-semantic baseline and are especially well suited for the recommendation of the types of citations they're conceptualized for.

%Citation recommendation systems, a possible remedy, 
%Researchers spent a considerable amount of time identifying publications that are worthwhile reading and appropriate to reference. The development of systems to aid in these tasks is an active area of research.
%Citations are the means by which scholars relate their research to existing work, 

\chapter*{Zusammenfassung}
fu bar
