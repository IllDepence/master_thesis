\chapter{Introduction}\label{chap:introduction}
Chapters are divided into sections.

\section{Example Section}\label{sec:ex}
Copypasta of useful stuff below.
\begin{itemize}
    \item Put a tilde (nbsp) in front of citations~\cite{Moravcsik1975}.
    \item \todo{Do this!}
    \item \extend{Write more when new results are out!}
    \item \draft{Hacky text!}
    \item \chapref{chap:introduction} % also \chapref{} \secref{sec:XY} \eqref{} \figref{} \tabref{}
    \item the colors of the Uni
    \begin{itemize}
        \item {\color{UniBlue}UniBlue}
        \item {\color{UniRed}UniRed}
        \item {\color{UniGrey}UniGrey}
    \end{itemize}
    \item a command for naming matrices $\mat{G}$, and naming vectors $\vec{a}$. This overwrites the default behavior of having an arrow over vectors, sticking to the naming conventions  normal font for scalars, bold-lowercase for vectors, and bold-uppercase for matrices.
    \item named equations:
        \begin{align}
            d(a,b) &= d(b,a)\\ \eqname{symmetry}
        \end{align}
    \item Use ``these'' for citing, not "these"
    \item If an equation is at the end of a sentence, add a full stop. If it's not the end, add a comma: {$a= b + c$~~~~(1),}
    \item \url{https://en.wikipedia.org}
    \item Do not overuse footnotes\footnote{\url{https://en.wikipedia.org}} if possible.
\end{itemize}
