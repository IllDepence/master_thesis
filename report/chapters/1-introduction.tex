\chapter{Introduction}\label{chap:introduction}
\section{Motivation}
Citations are a central building block of scholarly discourse. They are the means by which scholars relate their research to existing work---be it in backing up claims, criticising, naming examples or engaging in any other form. Citing in a meaningful way requires an author to be aware of publications relevant to their work.
Here, the ever increasing amount of new reseach publications per year poses a serious challenge. Even with academic search engines like Goolge Scholar and CiteSeerX at our disposal, identifying publications that are worthwhile to examine and appropriate to reference remains a time consuming task.

It is therefore not suprising that methods to aid researchers in these tasks have been and still are being actively researched. While diverse in nature, the common core of these efforts is the goal to utilize the automated processing of publications. This can be achieved by either extracting information from publications as they are today~\cite{Nasar2018,Beel2016}, or by introducing explicit semantic representations to facilitate automated processing~\cite{BuckinghamShum2000,Schneider2013,Kitamoto2015}. Once processed, a typical method is to harvest human made citations, analyze them~\cite{Abujbara2013,Teufel2006a,Teufel2006b} and use them for example to recommend papers~\cite{Beel2016} or aid in document exploration~\cite{Berger2016}. Although systems like this have existed for over 20 years~\cite{Bollacker1998,Beel2016}, there is not a lot of work looking into the use of explicit semantic representations for the recommendation of papers. This is why this thesis will investiage their application. More specifically, we will concentrate on the task of recommending papers for citation---as opposed to, for example, discovery. What this encompasses will be described in more detail in the following section.

% Systems for recommending papers have existed since 1998~\cite{Bollacker1998,Beel2016}. The closely related field of citation analysis has an even longer history that spans multiple disciplices including applied linguistics, history and sociology of sience and information science~\cite{Swales1986,White2004}.

% also there’s different approaches to it (collab.fil. / content based fil.(input=paper / input=sentence / input=sentence+one citation\cite{Kobayashi2018} / input=abstract\cite{Ayala-Gomez2018}) / graph based)

\section{Problem setting}\label{sec:problemsetting}
The goal of this thesis is to assess the viability of and requirements for employing explicit semantic representations for the task of citation recommendation.

\section{Method}\label{sec:method}
generate a data set fit for investigation of citation recommendation employing named entities and claims

devise citation recommendation approach implement and evaluate

\section{Contributions}\label{sec:contributions}

\section{Document structure}\label{sec:documentstructure}
foo bar

\section{Example Section}\label{sec:ex}
Copypasta of useful stuff below.
\begin{itemize}
    \item Put a tilde (nbsp) in front of citations~\cite{Moravcsik1975}.
    \item \todo{Do this!}
    \item \extend{Write more when new results are out!}
    \item \draft{Hacky text!}
    \item \chapref{chap:introduction} % also \chapref{} \secref{sec:XY} \eqref{} \figref{} \tabref{}
    \item the colors of the Uni
    \begin{itemize}
        \item {\color{UniBlue}UniBlue}
        \item {\color{UniRed}UniRed}
        \item {\color{UniGrey}UniGrey}
    \end{itemize}
    \item a command for naming matrices $\mat{G}$, and naming vectors $\vec{a}$. This overwrites the default behavior of having an arrow over vectors, sticking to the naming conventions  normal font for scalars, bold-lowercase for vectors, and bold-uppercase for matrices.
    \item named equations:
        \begin{align}
            d(a,b) &= d(b,a)\\ \eqname{symmetry}
        \end{align}
    \item Use ``these'' for citing, not "these"
    \item If an equation is at the end of a sentence, add a full stop. If it's not the end, add a comma: {$a= b + c$~~~~(1),}
    \item \url{https://en.wikipedia.org}
    \item Do not overuse footnotes\footnote{\url{https://en.wikipedia.org}} if possible.
\end{itemize}
