\chapter{Introduction}\label{chap:introduction}
\section{Motivation}
Citations are a central building block of scholarly discourse. They are the means by which scholars relate their research to existing work---be it in backing up claims, criticising, naming examples or engaging in any other way. Citing in a meaningful way requires an author to be aware of publications relevant to their work.
Here, the ever increasing amount of new reseach publications per year poses a serious challenge. Even with academic search engines like Goolge Scholar and CiteSeerX at our disposal, identifying publications that are worthwhile to examine and appropriate to reference remains a time consuming task.

It is therefore not suprising that methods to aid researchers in these tasks have been and still are being actively researched. 

While systems for recommending papers have existed since 1998~\cite{Bollacker1998,Beel2016}. The closely related field of citation analysis has an even longer history that spans multiple disciplices including applied linguistics, history and sociology of sience and information science~\cite{Swales1986,White2004}.

It is therefore not suprising that efforts to aid researchers in these tasks are no novelty. Systems for recommending papers have existed since 1998~\cite{Bollacker1998,Beel2016}. The closely related field of citation analysis has an even longer history that spans multiple disciplices including applied linguistics, history and sociology of sience and information science~\cite{Swales1986,White2004}.

there's peripheral fields like citation context analysis\cite{HERNANDEZ-ALVAREZ2016}, using citations for document exploration\cite{Berger2016}, citation function\cite{Moravcsik1975,Abujbara2013,Teufel2006a,Teufel2006b}, citation based plagiarizm detection\cite{Gipp2010}

also there’s different approaches to it (collab.fil. / content based fil.(input=paper / input=sentence / input=sentence+one citation\cite{Kobayashi2018} / input=abstract\cite{Ayala-Gomez2018}) / graph based)

also there are efforts/ideas/... for using explicit semantic representations for scholarly discours\cite{BuckinghamShum2000,Schneider2013,Kitamoto2015}

therefore investigate how they can be applied to citation recommendation

\section{Problem setting}\label{sec:problemsetting}

\section{Method}\label{sec:method}
generate a data set fit for investigation of citation recommendation employing named entities and claims

devise citation recommendation approach implement and evaluate

\section{Document structure}\label{sec:documentstructure}
foo bar

\section{Example Section}\label{sec:ex}
Copypasta of useful stuff below.
\begin{itemize}
    \item Put a tilde (nbsp) in front of citations~\cite{Moravcsik1975}.
    \item \todo{Do this!}
    \item \extend{Write more when new results are out!}
    \item \draft{Hacky text!}
    \item \chapref{chap:introduction} % also \chapref{} \secref{sec:XY} \eqref{} \figref{} \tabref{}
    \item the colors of the Uni
    \begin{itemize}
        \item {\color{UniBlue}UniBlue}
        \item {\color{UniRed}UniRed}
        \item {\color{UniGrey}UniGrey}
    \end{itemize}
    \item a command for naming matrices $\mat{G}$, and naming vectors $\vec{a}$. This overwrites the default behavior of having an arrow over vectors, sticking to the naming conventions  normal font for scalars, bold-lowercase for vectors, and bold-uppercase for matrices.
    \item named equations:
        \begin{align}
            d(a,b) &= d(b,a)\\ \eqname{symmetry}
        \end{align}
    \item Use ``these'' for citing, not "these"
    \item If an equation is at the end of a sentence, add a full stop. If it's not the end, add a comma: {$a= b + c$~~~~(1),}
    \item \url{https://en.wikipedia.org}
    \item Do not overuse footnotes\footnote{\url{https://en.wikipedia.org}} if possible.
\end{itemize}
