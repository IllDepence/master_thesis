\chapter{Background}\label{chap:background}
In this chapter we will define important terms and lay out the theoretical background for the remainder of the thesis.

\section{Definition of terms}
\paragraph{Citation} The term \emph{``citation''} can refer to both the act of citing as well as the occurrence of being cited. This can be illustrated with the phrase \emph{``an author's citations''}. In \cite{Beel2016} Beel et al. write \emph{``McNee et al. assumed that an author's citations indicate a positive vote for a paper [93].''} (the author cites), while Myers~\cite{Myers1970} writes \emph{``Thus, the number of an author's citations, in this study, means the number of articles in which one or more of his publications were cited.''} (the author is being cited). Like the latter example we will make an effort to use the term unambiguosly.
% also: \cite{Beel2016} \emph{``a few articles gained many citations (the maximum was 528 citations for [43]) and many articles had few citations, see Fig. 3.''} -> passive
\paragraph{Citation marker} A citation marker is a string of text that identifies an entry within a document's reference section. Examples are ``[27]'', ``[Bol98]'' and ``(He, 2010)''. These are placed within the document whenever the author refers to one of the documents in the reference section.
\paragraph{Citation context} The context of a citation is the text surrounding its citation marker. Typical sizes are 1--3 sentences. The sentence containing the marker is also sometimes referred to as ``citing sentence''. 

\begin{table}
\centering
    \caption[Examples of citations and their categorization into integral/non-integral as well as syntactic/non-syntactic.]{Examples of citations and their categorization into integral/non-integral (values left of split) as well as syntactic/non-syntactic (values right of split).}
    \label{tab:integralsyntactic}
\begin{center}
    \begin{tabular}{llllll|ll}%m{8cm}
    \toprule
    Context excerpt (citation marker {\color{UniBlue}highlighted}) & \rotatebox{90}{Swales~\cite{Swales1990}} & \rotatebox{90}{Hyland~\cite{Hyland1999}} & \rotatebox{90}{Thompson~\cite{Thompson2001}} & \rotatebox{90}{Okamura~\cite{Okamura2008}} & \rotatebox{90}{Lamers et al.~\cite{Lamers2018}} & \rotatebox{90}{Whidby et al.~\cite{Whidby2011}} & \rotatebox{90}{Abujbara et al.~\cite{Abujbara2012}} \\
    \midrule
    ``Swales {\color{UniBlue}(1990)} has argued that ...''                 & i & i & i & i & i & n & ? \\
    ``{\color{UniBlue}Swales (1990)} has argued that ...''                 & i & i & i & i & n & s & s \\
    ``Swales {\color{UniBlue}[42]} has argued that ...''                   & i & i & i & i & i & n & n \\
    ``Swales has argued that ... {\color{UniBlue}[42]}''                   & i & i & i & i & i & n & n \\
    ``It has been argued {\color{UniBlue}(Swales, 1990)} that ...''         & n & n & n & n & n & n & n \\
    ``It has been argued {\color{UniBlue}[42]} that ...''                  & n & n & n & n & n & n & n \\
    ``According to {\color{UniBlue}(Swales, 1990)} it is ...'' & ? & ? & ? & ? & n & s & s \\
    ``According to {\color{UniBlue}[42]} it is ...''          & n & n & n & n & n & s & s \\
    ``... has been shown (see {\color{UniBlue}(Swales, 1990)}).''           & n & n & n & n & n & s & n \\
    \bottomrule
    \end{tabular}
\end{center}
\end{table}

\paragraph{Integral and syntactic citations} There are two somewhat similar, and at first glance easily confused notions condsidering a citation's role within its context. They are referred to as ``integral''---in the adjectival sense close in meaning to ``essential'' or ``inherent'', not what we denote in caluclus with $\int$---and ``syntactic''. Integral citations were first defined by Swales~\cite{Swales1990} in 1990 and are a frequently used~\cite{Hyland1999,Thompson2001,Okamura2008,Lamers2018} measure in discourse analysis. An integral citation is, in Swales' own words, \emph{``one in which the name of the researcher occurs in the actual citing sentence as some sentence-element''}. Thompson~\cite{Thompson2001} rephrases the definition as \emph{``citations that [...] play an explicit grammatical role within a sentence''}. While what Thompson refers to as ``citations'' might be confused with the notion of citation markers, the examples given in \cite{Thompson2001} clearly indicate that a ``citaion'' is to mean an author's name in their definition. The second notion, ``syntactic'' (as used in \cite{Whidby2011} and \cite{Abujbara2012}), is concerned with whether or not a \emph{citation marker} has a grammatical role within its context. In other words, if removing the citation marker would make the citing sentence ungrammatical, then it is syntactic. Table~\ref{tab:integralsyntactic} gives an overview of examples for both concepts. Note that Lamers et al.~\cite{Lamers2018} provide a classification algorithm for integral and non-integral citations that slightly differs from Swales' original definition depending on the interpretation of a citation marker's scope, but also gives a clear classification in an edge case where Swales's definition is unclear. Furthermore note that the the two ways for distinguishing syntactic and non-syntactic citations found in literature are not identical. This is in part because the method given in \cite{Abujbara2012} is kept rather simple. For the intents and purposes of this work we can follow the definitions of Lamers et al. and Whidby et al. for ``integral'' and ``syntactic'' respectively.
\paragraph{Reference} References are the entries in a document's reference section. Each reference should unambiguously identify another document that it points to. In the context of parsing reference sections we will, at times, refer to references as ``reference strings''. An example of a reference is \emph{``W. Huang, , P. Mitra, and C. L. Giles, `RefSeer: A citation recommendation system,' in IEEE/ACM Joint Conference on Digital Libraries, pp. 371–374, Sep. 2014.''}.
\paragraph{Reference resolution} The act of parsing a reference string and matching it to a document identifier is what we refer to as ``reference resolution''. Examples can be found in Section~\ref{sec:refresol}.
% \paragraph{Citation recommendation} global/local  % already explained in first chapter
\paragraph{Named Entity} In natural language processing the term ``Named Entity'' (NE) describes unambiguous abstract or physical entities which have a proper name. Examples are people (Tim Berners-Lee), places (the city Taipei), organizations (the Free Software Foundation) and concepts (Okapi BM25). For a more detailed discussion of the term see~\cite{Nadeau2007}.
\paragraph{Noun phrase} A noun phrase is a group of words (or a single word) functioning grammatically as one unit, that has a noun at its head. Examples are ``\emph{example}'', ``noun \emph{phrase}'' and ``context-based co-citation \emph{recommendation}'' (head of the phrase \emph{highlighted}).
\paragraph{Claim} For our purposes we define a ``claim'' as ...

\section{Theoretical background}
\subsection{Foo}
supervized ml (briefly)
% michael:
% - peculiarities of CiteRec (many papers, few users, even in the field of rec systems);
% - diverse nature of publications to be parsed (what data can be extracted from papers, ...);
% in general: Only focus on really, i.e., directly relevant items/concepts mentioned later in the thesis.
\subsection{Bar}
explain harvesting citation contexts and comparing aggregrates to input
\subsection{Evaluation metrics}
Recall, MAP, MRR, NDCG
