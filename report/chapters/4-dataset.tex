\chapter{Data set}\label{chap:dataset}
approach approach.

\section{Existing data sets}
and why a new one was necessary

\begin{table}[ht]
\begin{center}
    % \caption{Overview of existing data sets.}
    % \label{tab:existing-data-sets}
    \begin{tabular}{llllll}
    \toprule
    Data set & \#Papers & Cit. context & Disciplines & Full text & Ref. IDs \\
    \midrule
    arXiv CS    &  90K & 1 sentence & CS & yes & DBLP \\ % \cite{Faerber2018LREC}
    CiteSeerX /RefSeer  &  1M & 400 chars & all & no & no \\ % \cite{Caragea2014} / \cite{HuangWCMG15}
    PubMed Central OA\footnote{\url{https://www.ncbi.nlm.nih.gov/pmc/tools/openftlist/}} & 2.3M & extractable & Biomed./Life Sci. & yes & mixed \\
    Scholarly v2\footnote{\url{http://www.comp.nus.edu.sg/~sugiyama/SchPaperRecData.html}}  & 100K & extractable & CS & yes & no \\
    ACL-ARC  & 11k & extractable & CS/comp. ling. & yes & no \\ % \cite{Bird2008ACLARC}
    ACL-AAN  & 18k & extractable & CS/comp. ling. & yes & no  \\ % \cite{Radev2013}
    \bottomrule
    \end{tabular}
\end{center}
    \caption[Table caption]{\textbf{Table caption.} foo bar...\\}
    \label{tab:datasets}
\end{table}

MAG\cite{Sinha2015} (use/analysis: \cite{Herrmannova2016,Paszcza2016,Hug2017})

use of PMC OAS\cite{Gipp2015,Duma2016,Galke2018,Bhagavatula2018} (PMC OAS problems: \cite{Gipp2015})

\section{Data set creation}
foo

survey paper on extraction of meta data (author, year, ...) and classification of sentences (method, goal, ...) from publications\cite{Nasar2018}

evaluation of reference string parsers\cite{Tkaczyk2018}, a dataset for reference string parsing\cite{Anzaroot2013}

\section{Data set evaluation}
bar
