\chapter{Approach}\label{chap:approach}
In order to investigate the use of explicit semantic representations for the task of local citation recommendation we first need to decide which kinds of semantic constructs we want to model. As a starting point for this we looked into the field of citation context analysis~\cite{HERNANDEZ-ALVAREZ2016}. A common task in this area is the classification of citation contexts by their polarity (positive/neutral/negative) and function (often based on the four dimensions identified by Moravcsik et al.~\cite{Moravcsik1975}, conceptual/operational, evolutionary/juxtapositional, organic/perfunctory, conformative/negational). Such approaches are primarily concerned with the \emph{intent} of the author rather than the \emph{content} of what is being cited. We can therefore not expect to derive types of semantic constructs directly from citation functions. Starting from an established typology of citation functions will, however, ensure that we consider a wide range of different citations rather than cherry picking those that fit our preconceptions.

\begin{table}[tb]
\centering
    \caption{Semantic constructs in citation contexts from a range of citation functions used in the field of citation context analysis.}
    \label{tab:citfunctions}
\begin{center}
    \begin{tabular}{m{2.7cm}lm{8.5cm}}
    \toprule
    Function & Construct & Examples (semantic construct \emph{highlighted})\\
    \midrule
    Attribution & claim & ``Berners-Lee et al.~\cite{Berners-Lee2001} argue that \emph{structured collections of information and sets of inference rules are prerequesites for the semantic web to function}.'' \\\noalign{\smallskip}
    \  & NE & ``A variation of this task is `\emph{context-based co-citation recommendation}'~\cite{Kobayashi2018}.'' \\\noalign{\smallskip}
    \  & - & ``In \cite{Duma2014} Duma et al. test the effectiveness of using a variety of document internal and external text inputs to a TF-IDF model.'' \\\noalign{\medskip}
    Exemplification & NE & ``We looked into approaches to \emph{local citation recommendation} such as~\cite{He2010,Huang2014,Huang2015,Duma2014,Duma2016,Ebesu2017,Kobayashi2018,Jeong2019} for our investigation.'' \\\noalign{\medskip}
    Further reference & - & ``See \cite{Niklaus2018} for a comprehensive overview.'' \\\noalign{\medskip}
    Statement of use & NE & ``We use \emph{CiteSeerX}~\cite{Caragea2014} for our evaluation.'' \\\noalign{\medskip}
    Application & NE & ``Using this mechanism we perform `\emph{context-based co-citation recommendation}'~\cite{Kobayashi2018}.'' \\\noalign{\medskip}
    Evaluation & - & ``The use of DBLP in \cite{Faerber2018} restricts their data set to the field of computer science.'' \\\noalign{\medskip}
    Establishing links between sources& claim & ``A common motivation brought forward for research on citation recommendation is that \emph{finding proper citations is a time consuming task} \cite{He2010,He2011,Ebesu2017,Kobayashi2018}.'' \\\noalign{\smallskip}
    \  & - & ``Lamers et al.~\cite{Lamers2018} base their definition on the author's name whereas Thompson~\cite{Thompson2001} focusses on the grammatical role of the citation marker.'' \\\noalign{\medskip}
    Comparison of own work with sources& claim & ``Like \cite{Faerber2018} we find that, albeit written in a structured language, \emph{parsing \LaTeX{} sources is a non trivial task}.'' \\
    \bottomrule
    \end{tabular}
\end{center}
\end{table}

Table~\ref{tab:citfunctions} lists categories of citation functions along with the kinds of semantic constructs that can be found in such citation contexts. The list of citation functions is taken from \cite{Petric2007} (and therein built upon \cite{Thompson2001}). This study was selected because it gives an overview of previous attempts to classify citations and presents their new typology with extensive explanation as well as example contexts. Examining contexts from each of the eight functions we identify two types of semantic constructs: named entities (NE) and claims (or statements). The rationale behind these two is as follows. Named entities can identify reference publications for a certain data set/tool/concept (see \emph{Attribution}, \emph{Statement of use} and \emph{Application} in Table~\ref{tab:citfunctions}) as well as a method/field of study common to a selection of publications (see \emph{Exemplification} in Table~\ref{tab:citfunctions}). Claims can identify publications that can be cited to back or support the very claim contained in a citation context. Note that, considering a larger citation context, argumentative structures could also be considered viable semantic constructs for local citation recommendation. To keep the scope of this thesis at a reasonable level we will limit our investigation to named entities and claims.

The following sections will describe our investigation of entity based and claim based models for local citation recommendation.

\section{Entity based recommendation}
The intuition behind an entity based approach is, that there exists a reference publication for a named entity. Examples would be a data set (``CiteSeerX~\cite{Caragea2014}''), a tool (``Neural ParsCit~\cite{Animesh2018}'') or a concept (``Semantic Web~\cite{Berners-Lee2001}''). In a more loose sense this can also include publications being referred to as examples (``approaches to local citation recommendation~\cite{He2010,Huang2014,Huang2015,Duma2014,Duma2016,Ebesu2017,Kobayashi2018,Jeong2019}'').

first DBpedia Spotlight (only mention in passing), then MAG FoS, then NPs

\subsection{Fields of study in the MAG}

\subsection{Noun phrases}

\section{Claim based recommendation}

how citations are embedded in sentences (integral/non-integral\cite{Swales1990,Hyland1999,Thompson2001,Okamura2008,Lamers2018})

\subsection{Tools for extracting claims}
tools tools

also: Survey on open information extraction\cite{Niklaus2018}

context specific claim detection\cite{Levy2014}

if only papers where semantically annotated as proposed in \cite{BuckinghamShum2000}
\subsection{A predicate-argument model}
predpatt\cite{White2016,Zhang2017}

unfeasibility of use of PredPatt output as is

loosened predicate:parameter model

predicates could be grouped/clustered to represent functions as in \cite{Gabor2018}

alternative view: model gives a selective citation context derived from claim structure (cf. concept of reference scope as sub part of citation context sentence\cite{Abujbara2012,RAHUL2017}
