\chapter{Conclusion}\label{chap:conclusion}
%Researchers spent a considerable amount of time identifying publications that are worthwhile reading and appropriate to reference. The development of systems to aid in these tasks is an active area of research.
In the field of local citation recommendation, the explicit semantic modelling of citation contexts is not well explored yet. In order to investigate the merit of such approaches, we generated a new data set from arXiv \LaTeX{} sources with accurate and precise citation data as well as full text paper contents. Using this data set we developed semantic models of citation contexts based on entities as well as claim structures. We then evaluated our models on several data sets in a citation re-prediction setting and furthermore conducted a small user study. One of the entity based models, NPmarker, which captures noun phrases preceding the citation marker, performs best at low cut-offs and in the MRR metric. Low cut-offs and measuring the MRR can be interpreted as emulating citations for reference publications. This interpretation is also backed by the results of the user study, where NPmarker outperformed all other models when recommending for citation contexts, that referenced a NE or concept. We therefore conclude that NPmarker is well suited for recommending such types of citations. Our claim based model on its own does not compare in performance to a BoW baseline, but outperforms aforementioned when combined with it (Claim+BoW). We take this as an indication that the model encodes important information which the non-semantic BoW model is not able to capture. In the user study Claim+BoW performs best for citation contexts, in which a claim is backed by the target citation. This suggests that the model indeed captures information related to claim structures.

Overall, we note that when considering citations \emph{in general}, it is far from trivial to outperform tried and tested information retrieval techniques like a TFIDF weighted BoW model. This may not be that surprising though, as citations come in many forms, and for each of them different aspects of the citation context are important. Significant improvements in citation recommendation might therefore require taking into account these differences, for example through the application of specialized models preceded by a classification step. The evaluation of our models indicates their suitability for recommending certain types of citations. The further development of these models and research on the classification of citations are therefore a promising direction of pursuit.
