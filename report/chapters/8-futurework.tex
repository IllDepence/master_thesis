\chapter{Future work}\label{chap:todo}
data set:
arXiv keeps growing, therefore data set can grow,
btw. data set is already extended with 2018 data\cite{Saier2019}

also for data set, look into including formula info instead of using replacement token (as important knowledge is contained in formulas, especially in math and physics publications). (math retrieval based on arXiv LaTeX: \cite{Aizawa2014,Zanibbi2016})

segue:
b/c data set spans several FoS,
test different FoS of arXiv data (while keeping data composition comparable, b/c as seen in eval this has a significant effect on evaluation outcomes)

models:
NPmarker so far only tested on arXiv data (b/c marker position)
for data where no precise citation marker can be guaraneed, look into heuristically identifying marker position

claim model
predicates could be grouped/clustered to represent functions as in \cite{Gabor2018}
+
b/c claim model is predicate-argument, it might allow for semantic/some kind of enhanced user search (e.g. showing citations backing claims that sth. \emph{is:NP-hard})

marker-aware claim model when handling of non-syntactic citations is solved

in general:
As a first step identify types of citations more systematically.
For different types, different models.

far future:
assessing credibility of claims\cite{Popat2016}
Argumentative structures. (Argumentation mining\cite{Stab2016,Lippi2016,Habernal2017})
